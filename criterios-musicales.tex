\subsection*{Criterios de edición musical}
\noindent Todos los tonos se han transcrito a claves modernas. Aquellos escritos en claves altas (clave de Sol para los Tiple) se han transpuesto una cuarta justa hacia abajo adaptando la armadura. Tanto la infomación de las claves usadas en el original como la transposición si la hubiera se indican en los datos musicales de cada tono.

\noindent Las sílabas que se cantan sobre más de una nota, que en el original se indican mediante una ligadura o semimínimas agruppadas, se indican en la transcripción siempre con una ligadura entre las notas y un guión bajo continuo en el texto de cada una de las notas posteriores a la primera. 

\noindent Las alteraciones accidentales añadidas por semitonia subintelecta se indican entre paréntesis. Se han añadido también accidentales cuando la nota mantiene la alteración del compás anterior. Cualquier otro accidental añadido se indica en las notas críticas.

\noindent La numeración de los compases de inicia desde 1 en cada sección.

\noindent Las expresiones dinámicas se han modernizado: pp por "brando".


Finalmente, se han incluido dos versiones de la partitura, una con todas las coplas debidamente alineadas con cada una de las voces y otra que solo incluye la primera copla, mostrando el resto al final.

\todo{decidir si presentar una opcion de partitura "musicologica" donde las anotaciones críticas se incluyan también en la partitura}

%TODO: añadir el resto

\noindent El compás ternario C3 se ha transcrito a 3/2 sin reducir los valores de las notas. Esto habrá de tenerse en cuenta de cara al tempo cuando en una misma sección se produzca un cambio a compás binario.

