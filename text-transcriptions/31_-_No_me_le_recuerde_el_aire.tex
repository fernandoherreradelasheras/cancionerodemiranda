\documentclass[titlepage,hidelinks]{article}
\usepackage[T1]{fontenc} 
\usepackage[utf8]{inputenc} 
\usepackage{multirow} 
\usepackage{tabularx} 
\usepackage{color} 
\usepackage{textcomp} 
\usepackage{tipa}
\usepackage{amsmath} 
\usepackage{amssymb} 
\usepackage{amsfonts} 
\usepackage{amsxtra} 
\usepackage{isomath} 
\usepackage{mathtools} 
\usepackage{txfonts} 
\usepackage{upgreek} 
\usepackage{enumerate} 
\usepackage{tensor} 
\usepackage{pifont} 
\usepackage{ulem} 
\usepackage{calc,xfrac}
\usepackage{arydshln} 
\usepackage[english]{babel}
\usepackage{subcaption}
\usepackage{pdfpages}
\usepackage{verse}
\usepackage{hyperref} 
\usepackage{multicol}

\providecommand{\tightlist}{%
  \setlength{\itemsep}{0pt}\setlength{\parskip}{0pt}}


\DeclareCaptionFormat{custom}
{%
	\textbf{#3}
}
\captionsetup{format=custom,justification=raggedleft,font={Large}}
\hypersetup{
    colorlinks=true,
    linkcolor=cyan,
    urlcolor=blue,
  }


\date{}
\input{values.tex}

\begin{document}
\begin{titlepage} % Suppresses headers and footers on the title page
  \vspace*{\baselineskip} % White space at the top of the page
  \rule{\textwidth}{1.6pt}\vspace*{-\baselineskip}\vspace*{2pt} % Thick horizontal rule
  \rule{\textwidth}{0.4pt} % Thin horizontal rule
  \vspace{0.75\baselineskip} % Whitespace above the title
  \mytitle
  \vspace{0.75\baselineskip} % Whitespace below the title
  \rule{\textwidth}{0.4pt}\vspace*{-\baselineskip}\vspace{3.2pt} % Thin horizontal rule
  \rule{\textwidth}{1.6pt} % Thick horizontal rule
  \vspace{5\baselineskip} % Whitespace after the title block

{\Large Texto: \mytext

Música: \mymusic

\vspace{10\baselineskip} % Whitespace after the title block
Versión: \myversion \myversioncomment \\}
\end{titlepage}





\centering
\Large{

No me le recuerde el aire,

al amor que está dormido,

que cuando está el sol en calma

no soplan los airecillos.
\vspace{2\baselineskip}

Parece el nevado cierzo,

porque está desnudo el niño

y a un jazmín el maltratarle

no\footnote{\textsuperscript{}El ``no'' falta en superius primus} es airoso desatino.
\vspace{2\baselineskip}

Dejen que el amor descanse,

parece el céfiro frío

que a un aire leve despierta

un cuidadoso dormido.
\vspace{2\baselineskip}

Véase cómo enfrenaron,

los cristales sus ruidos:

no cruzan los arboledos

ni cantan los pajarillos.

Cómo se duermen las flores,

cómo se callan los riscos,

cómo el silencio reposa,

cómo el mar está tranquilo
\vspace{2\baselineskip}

Todo le preste silencio,

que es compasivo el \footnote{\textsuperscript{}``al'' en superius primus}cariño

pues de llorar fatigado,

se duerme el dulce Cupido.

\vspace{4\baselineskip}
\textit{Estribillo:}
\vspace{1\baselineskip}

Airecillos mansos,

paso, quedito,

que duerme mi niño

y le bastan los aires

de sus suspiros.
\input{figures.tex}
\end{document}
