\documentclass[titlepage,hidelinks]{article}
\usepackage[T1]{fontenc} 
\usepackage[utf8]{inputenc} 
\usepackage{multirow} 
\usepackage{tabularx} 
\usepackage{color} 
\usepackage{textcomp} 
\usepackage{tipa}
\usepackage{amsmath} 
\usepackage{amssymb} 
\usepackage{amsfonts} 
\usepackage{amsxtra} 
\usepackage{isomath} 
\usepackage{mathtools} 
\usepackage{txfonts} 
\usepackage{upgreek} 
\usepackage{enumerate} 
\usepackage{tensor} 
\usepackage{pifont} 
\usepackage{ulem} 
\usepackage{calc,xfrac}
\usepackage{arydshln} 
\usepackage[english]{babel}
\usepackage{subcaption}
\usepackage{pdfpages}
\usepackage{verse}
\usepackage{hyperref} 
\usepackage{multicol}

\providecommand{\tightlist}{%
  \setlength{\itemsep}{0pt}\setlength{\parskip}{0pt}}


\DeclareCaptionFormat{custom}
{%
	\textbf{#3}
}
\captionsetup{format=custom,justification=raggedleft,font={Large}}
\hypersetup{
    colorlinks=true,
    linkcolor=cyan,
    urlcolor=blue,
  }


\date{}
\input{values.tex}

\begin{document}
\begin{titlepage} % Suppresses headers and footers on the title page
  \vspace*{\baselineskip} % White space at the top of the page
  \rule{\textwidth}{1.6pt}\vspace*{-\baselineskip}\vspace*{2pt} % Thick horizontal rule
  \rule{\textwidth}{0.4pt} % Thin horizontal rule
  \vspace{0.75\baselineskip} % Whitespace above the title
  \mytitle
  \vspace{0.75\baselineskip} % Whitespace below the title
  \rule{\textwidth}{0.4pt}\vspace*{-\baselineskip}\vspace{3.2pt} % Thin horizontal rule
  \rule{\textwidth}{1.6pt} % Thick horizontal rule
  \vspace{5\baselineskip} % Whitespace after the title block

{\Large Texto: \mytext

Música: \mymusic

\vspace{10\baselineskip} % Whitespace after the title block
Versión: \myversion \myversioncomment \\}
\end{titlepage}





\centering
\Large{

Qué valentones, que campan

los lindos pies, de Jacinta

cuando con belleza andante

retoma toda la villa.
\vspace{2\baselineskip}

No hay ventura, difícil

a sus dos terzas, cuchillas

pues al alma más yzenta,

a{ú}n envainadas castigan.
\vspace{2\baselineskip}

Si los saca con desgarro

porque lo descubre, indicia

en el nacar de sus medias,

la sangre de mis heridas.
\vspace{2\baselineskip}

Su candidez es, su crimen

y obscura cárcel, habitan

donde presos por pequeños,

no les vale su justicia.
\vspace{2\baselineskip}

Si al ver se econden, y burlan

de la atención, más prevista

que como son tan chiquitos,

andan siempre en niñerías.

\vspace{4\baselineskip}
\textit{Estribillo:}
\vspace{1\baselineskip}

Huid zagales, de encontrar a Jacinta.

Condiches de azabache, y azules cintas,

porque dan cuchilladas con valentía

a las más libres almas siempre acuchillan.
\input{figures.tex}
\end{document}
