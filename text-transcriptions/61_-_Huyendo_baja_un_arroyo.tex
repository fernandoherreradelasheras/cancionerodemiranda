\centering
\Large{

Huyendo baja un arroyo

de la aspereza de un monte,

suspendiéndose en un valle

lisonjeado de sus flores.
\vspace{2\baselineskip}

Alegre y agradecido,

presurosamente corre,

porque siempre los halagos

son dulcísimas prisiones.
\vspace{2\baselineskip}

Aquí\footnote{\textsuperscript{}Esta estrofa no figura en la versión del LDTH, que por el contrario tiene 5 estrofas que no aparecen aquí} un día cuando el alba

con celestiales colores

de la noche, fugitiva,

desterró las sombras torpes.
\vspace{2\baselineskip}

En la cabaña de Filis,

Lucinda los ojos pone,

porque, de nuevo, rendidos,

la veneren y la adoren.

\vspace{4\baselineskip}
\textit{Estribillo:}
\vspace{1\baselineskip}

No hay consuelo, pastores,

para mi dolor,

que al salir el alba

se me puso el sol
