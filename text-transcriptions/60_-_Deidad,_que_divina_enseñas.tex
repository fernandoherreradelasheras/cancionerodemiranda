\centering
\Large{

Deidad, que divina enseñas,

por lo apacible y sereno

cuando es bien que llore el sol

cuando es bien que ría el cielo.
\vspace{2\baselineskip}

Como no han de abrasar todo,

esos lindos ojos negros,

si nunca les falta el aire

y siempre les sobra el fuego.
\vspace{2\baselineskip}

Quien oyendo tus palabras,

no reconoce que siendo,

dichas sin cuidado todas,

todas encierran misterios.
\vspace{2\baselineskip}

De lo fragante y lucido,

de tu peregrino aseo,

tienen las flores envidia,

tienen las estrellas celos.
\vspace{2\baselineskip}

Quando a dar al prado sales,

mil glorias con un paseo,

por seguirte a pie se olvidan

de volar los pensamientos.
\vspace{2\baselineskip}

Esto las voces decían,

de unos ruiseñores

a la divina lisonja,

de los males sin remedio.

\vspace{4\baselineskip}
\textit{Estribillo:}
\vspace{1\baselineskip}

Ruiseñores que libres cantáis

cantad no paréis

pues tanta dicha lográis

que sale a escucharos

la aurora que veis,

ay, ay, de quien es solo su canto,

presiones, suspiros, memorias y llanto
