\documentclass[titlepage,hidelinks]{article}
\usepackage[T1]{fontenc} 
\usepackage[utf8]{inputenc} 
\usepackage{multirow} 
\usepackage{tabularx} 
\usepackage{color} 
\usepackage{textcomp} 
\usepackage{tipa}
\usepackage{amsmath} 
\usepackage{amssymb} 
\usepackage{amsfonts} 
\usepackage{amsxtra} 
\usepackage{isomath} 
\usepackage{mathtools} 
\usepackage{txfonts} 
\usepackage{upgreek} 
\usepackage{enumerate} 
\usepackage{tensor} 
\usepackage{pifont} 
\usepackage{ulem} 
\usepackage{calc,xfrac}
\usepackage{arydshln} 
\usepackage[english]{babel}
\usepackage{subcaption}
\usepackage{pdfpages}
\usepackage{verse}
\usepackage{hyperref} 
\providecommand{\tightlist}{%
  \setlength{\itemsep}{0pt}\setlength{\parskip}{0pt}}


\DeclareCaptionFormat{custom}
{%
	\textbf{#3}
}
\captionsetup{format=custom,justification=raggedleft,font={Large}}
\hypersetup{
    colorlinks=true,
    linkcolor=cyan,
    urlcolor=blue,
  }


\date{}
\input{values.tex}

\begin{document}
\begin{titlepage} % Suppresses headers and footers on the title page
	\begin{center}
  \vspace*{\baselineskip} % White space at the top of the page
  \rule{\textwidth}{1.6pt}\vspace*{-\baselineskip}\vspace*{2pt} % Thick horizontal rule
  \rule{\textwidth}{0.4pt} % Thin horizontal rule
  \vspace{0.75\baselineskip} % Whitespace above the title
  \mytitle
  \vspace{0.75\baselineskip} % Whitespace below the title
  \rule{\textwidth}{0.4pt}\vspace*{-\baselineskip}\vspace{3.2pt} % Thin horizontal rule
  \rule{\textwidth}{1.6pt} % Thick horizontal rule
  \vspace{5\baselineskip} % Whitespace after the title block

{\Large Texto: \mytext 

Música: \mymusic}
\end{center}

\vspace{10\baselineskip} % Whitespace after the title block
\begin{flushleft}
Versión: \myversion

\textbf{Transcripción del texto:} \mytexttranscription

\textbf{Revisión del texto:} \mytextproofreading

\textbf{Validación del texto:} \mytextvalidation

\textbf{Transcripción de la música:} \mymusictranscription

\textbf{Revisión de la música:} \mymusicproofreading

\textbf{Validación de la música:} \mymusicvalidation

\textbf{Estudio poético:} \mypoeticstudy

\textbf{Estudio musical:} \mymusicalstudy 
\end{flushleft}
\end{titlepage}





\centering
\Large{

Coplas 1as:

Si al bosque la corte va,

ah que el bosque corte sea

venga la casa a la aldea

pues bosque la corte es ya,

sin luz y sin flor está

porque su luz superior,

se va con tanto rigor

que no hay quien tenerla pueda,

ay, se muda, que ay, ay queda

la aldea en su ausencia,

sin luz y sin flor.
\vspace{2\baselineskip}

Coplas 2as:

Aldea si te murieras,

en tal ausencia acertarás,

pues muriendo no envidiarás,

la ventura de las fieras

y tus flores lisonjeras,

siguiendo tu resplandor,

se van dejando un dolor,

que no hay mal al que no exceda,

ay se muda, que ay ay, se queda

la aldea en su ausencia sin luz y sin flor.
\vspace{2\baselineskip}

Quedarás aldea buena,

en manos de una memoria

preguntando por tus glorias

al efecto de tu pena,

t{ú} de mil saudades llena

tu sol hecho cazador,

calla que en el desamor

también la fortuna rueda

ay se muda, que ay ay, se queda

la aldea en su ausencia sin luz y sin flor.

\vspace{4\baselineskip}
\textit{Estribillo:}
\vspace{1\baselineskip}

Por hacer apacibles los bosques

se van a los bosques las flores

y el sol que ay se muda

la aldea en su ausencia sin luz

y sin flor
\input{figures.tex}
\end{document}
