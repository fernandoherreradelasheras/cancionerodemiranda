\documentclass[titlepage,hidelinks]{article}
\usepackage[T1]{fontenc} 
\usepackage[utf8]{inputenc} 
\usepackage{multirow} 
\usepackage{tabularx} 
\usepackage{color} 
\usepackage{textcomp} 
\usepackage{tipa}
\usepackage{amsmath} 
\usepackage{amssymb} 
\usepackage{amsfonts} 
\usepackage{amsxtra} 
\usepackage{isomath} 
\usepackage{mathtools} 
\usepackage{txfonts} 
\usepackage{upgreek} 
\usepackage{enumerate} 
\usepackage{tensor} 
\usepackage{pifont} 
\usepackage{ulem} 
\usepackage{calc,xfrac}
\usepackage{arydshln} 
\usepackage[english]{babel}
\usepackage{subcaption}
\usepackage{pdfpages}
\usepackage{verse}
\usepackage{hyperref} 
\providecommand{\tightlist}{%
  \setlength{\itemsep}{0pt}\setlength{\parskip}{0pt}}


\DeclareCaptionFormat{custom}
{%
	\textbf{#3}
}
\captionsetup{format=custom,justification=raggedleft,font={Large}}
\hypersetup{
    colorlinks=true,
    linkcolor=cyan,
    urlcolor=blue,
  }


\date{}
\input{values.tex}

\begin{document}
\begin{titlepage} % Suppresses headers and footers on the title page
	\begin{center}
  \vspace*{\baselineskip} % White space at the top of the page
  \rule{\textwidth}{1.6pt}\vspace*{-\baselineskip}\vspace*{2pt} % Thick horizontal rule
  \rule{\textwidth}{0.4pt} % Thin horizontal rule
  \vspace{0.75\baselineskip} % Whitespace above the title
  \mytitle
  \vspace{0.75\baselineskip} % Whitespace below the title
  \rule{\textwidth}{0.4pt}\vspace*{-\baselineskip}\vspace{3.2pt} % Thin horizontal rule
  \rule{\textwidth}{1.6pt} % Thick horizontal rule
  \vspace{5\baselineskip} % Whitespace after the title block

{\Large Texto: \mytext 

Música: \mymusic}
\end{center}

\vspace{10\baselineskip} % Whitespace after the title block
\begin{flushleft}
Versión: \myversion

\textbf{Transcripción del texto:} \mytexttranscription

\textbf{Revisión del texto:} \mytextproofreading

\textbf{Validación del texto:} \mytextvalidation

\textbf{Transcripción de la música:} \mymusictranscription

\textbf{Revisión de la música:} \mymusicproofreading

\textbf{Validación de la música:} \mymusicvalidation

\textbf{Estudio poético:} \mypoeticstudy

\textbf{Estudio musical:} \mymusicalstudy 
\end{flushleft}
\end{titlepage}





\centering
\Large{

Deidad, que divina enseñas,

por lo apacible y sereno

cuando es bien que llore el sol

cuando es bien que ría el cielo.
\vspace{2\baselineskip}

Como no han de abrasar todo,

esos lindos ojos negros,

si nunca les falta el aire

y siempre les sobra el fuego.
\vspace{2\baselineskip}

Quien oyendo tus palabras,

no reconoce que siendo,

dichas sin cuidado todas,

todas encierran misterios.
\vspace{2\baselineskip}

De lo fragante y lucido,

de tu peregrino aseo,

tienen las flores envidia,

tienen las estrellas celos.
\vspace{2\baselineskip}

Quando a dar al prado sales,

mil glorias con un paseo,

por seguirte a pie se olvidan

de volar los pensamientos.
\vspace{2\baselineskip}

Esto las voces decían,

de unos ruiseñores

a la divina lisonja,

de los males sin remedio.

\vspace{4\baselineskip}
\textit{Estribillo:}
\vspace{1\baselineskip}

Ruiseñores que libres cantáis

cantad no paréis

pues tanta dicha lográis

que sale a escucharos

la aurora que veis,

ay, ay, de quien es solo su canto,

presiones, suspiros, memorias y llanto
\input{figures.tex}
\end{document}
