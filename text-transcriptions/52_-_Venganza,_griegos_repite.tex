\centering
\Large{

Venganza, griegos repite

Aquiles, blasón de todos,

blandiendo un rayo por asta

y desbocando un escollo.
\vspace{2\baselineskip}

Arda Helena, y arda Paris,

prosigue el caudillo heroico

siendo en delitos de celos,

complice el mármol y el plomo.
\vspace{2\baselineskip}

Y al fuego encendiendo el aire,

al son de gemidos roncos

con su elemento compite,

que está menos poderoso\footnote{\textsuperscript{}Las coplas 4ª y 6ª de Onteniente f.25v no se incluyen aquí}.
\vspace{2\baselineskip}

Aquel pájaro difunto,

fénix del oriente hermoso

sin renacer en sus llamas,

jeroglífico del otro.

\vspace{4\baselineskip}
\textit{Estribillo:}
\vspace{1\baselineskip}

{\textexclamdown}Piedad, favor, socorro!,

y es pedir puerto al golfo

pensar\footnote{\textsuperscript{}A partir de este verso el estribillo difiere de Onteniente f.25 y de CPMHL-23} que ha de vencer el mundo

sintiendo mis despojos

sus rayos pompas y blasones locos
