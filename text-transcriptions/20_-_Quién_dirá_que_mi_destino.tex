\centering
\Large{

Quién dirá que mi destino,

me lleva tras un desdén,

sin que merézcalo sino

porque no obliga el querer.
\vspace{2\baselineskip}

Quién dirá que puede un alma,

contrastar tal esquivez,

pues no acaba de un desprecio,

y de anima en lo cruel.
\vspace{2\baselineskip}

Quién dirá que no es fineza,

seguir amante, una fe,

sin más premio que la dicha

de adorar sin merecer.
\vspace{2\baselineskip}

Quién dirá que no es la muerte,

de morir por querer bien,

vida de un amor que solo

el morir su vida es.
\vspace{2\baselineskip}

Quién dirá que la esperanza,

quita a la fineza el ser,

que amor que espera envidioso,

no es amor, es interés\footnote{\textsuperscript{}Superior secundus no tiene las dos últimas estrofas}.
\vspace{2\baselineskip}

Quién dirá que puede tanto,

afecto que no se ve,

siendo antes incendio a voces

y aspid oculto después.
