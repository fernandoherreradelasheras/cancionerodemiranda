\documentclass[titlepage,hidelinks]{article}
\usepackage[T1]{fontenc} 
\usepackage[utf8]{inputenc} 
\usepackage{multirow} 
\usepackage{tabularx} 
\usepackage{color} 
\usepackage{textcomp} 
\usepackage{tipa}
\usepackage{amsmath} 
\usepackage{amssymb} 
\usepackage{amsfonts} 
\usepackage{amsxtra} 
\usepackage{isomath} 
\usepackage{mathtools} 
\usepackage{txfonts} 
\usepackage{upgreek} 
\usepackage{enumerate} 
\usepackage{tensor} 
\usepackage{pifont} 
\usepackage{ulem} 
\usepackage{calc,xfrac}
\usepackage{arydshln} 
\usepackage[english]{babel}
\usepackage{subcaption}
\usepackage{pdfpages}
\usepackage{verse}
\usepackage{hyperref} 
\providecommand{\tightlist}{%
  \setlength{\itemsep}{0pt}\setlength{\parskip}{0pt}}


\DeclareCaptionFormat{custom}
{%
	\textbf{#3}
}
\captionsetup{format=custom,justification=raggedleft,font={Large}}
\hypersetup{
    colorlinks=true,
    linkcolor=cyan,
    urlcolor=blue,
  }


\date{}
\input{values.tex}

\begin{document}
\begin{titlepage} % Suppresses headers and footers on the title page
	\begin{center}
  \vspace*{\baselineskip} % White space at the top of the page
  \rule{\textwidth}{1.6pt}\vspace*{-\baselineskip}\vspace*{2pt} % Thick horizontal rule
  \rule{\textwidth}{0.4pt} % Thin horizontal rule
  \vspace{0.75\baselineskip} % Whitespace above the title
  \mytitle
  \vspace{0.75\baselineskip} % Whitespace below the title
  \rule{\textwidth}{0.4pt}\vspace*{-\baselineskip}\vspace{3.2pt} % Thin horizontal rule
  \rule{\textwidth}{1.6pt} % Thick horizontal rule
  \vspace{5\baselineskip} % Whitespace after the title block

{\Large Texto: \mytext 

Música: \mymusic}
\end{center}

\vspace{10\baselineskip} % Whitespace after the title block
\begin{flushleft}
Versión: \myversion

\textbf{Transcripción del texto:} \mytexttranscription

\textbf{Revisión del texto:} \mytextproofreading

\textbf{Validación del texto:} \mytextvalidation

\textbf{Transcripción de la música:} \mymusictranscription

\textbf{Revisión de la música:} \mymusicproofreading

\textbf{Validación de la música:} \mymusicvalidation

\textbf{Estudio poético:} \mypoeticstudy

\textbf{Estudio musical:} \mymusicalstudy 
\end{flushleft}
\end{titlepage}





\centering
\Large{

{[}a 4{]}

Esc{ú}chame Fabio,

esc{ú}chame ahora,

que está el aire ciego,

y la noche sorda.
\vspace{2\baselineskip}

El ganado duerme,

el céfiro sopla,

esperanzas mías

pues suyas son todas.
\vspace{2\baselineskip}

{[}solo s1{]}

Íbame yo al soto,

que el mayo le entolda,

su primero día de aljófar\footnote{\textsuperscript{}``adelfas'' en el original} y rosas
\vspace{2\baselineskip}

{[}solo a{]}

{[} estatua la villa,

  muy cerca de toda,

  lejos de mí misma,

  (mas consigo propia) {]}
\vspace{2\baselineskip}

{[}solo s2{]}

Quando miro, ay triste

en una carroza,

al sol que allí estaba

cerca con las ondas.
\vspace{2\baselineskip}

{[}solo t{]}

Era el sol Antandra,

que entre mil pastoras

muy acompañada

estaba muy sola.
\vspace{2\baselineskip}

{[}solo s1{]}

Eranle los ojos

del color y forma,

cuales la ventura,

del que los adora.
\vspace{2\baselineskip}

{[}solo a{]}

{[} Has visto, (no has visto,

  pues vives) la boca,

  donde el viento se peina,

  con dientes de aljofar. {]}
\vspace{2\baselineskip}

{]}

{[}solo s2{]}

De tu frente el alba,

los jazmines roba,

nunca le echan menos,

porque siempre sobran.
\vspace{2\baselineskip}

{[}solo t{]}

Quien miró su cuello

dice que se nombra

cristal de garganta,

no cristal de roca.
\vspace{2\baselineskip}

{[}solo s1{]}

Hasta ver sus manos,

tuvo vana gloria

de limpia la nieve

de blanca la aurora.
\vspace{2\baselineskip}

{[}solo a{]}

{[} expiraba el día,

  y porque las horas

  más y más se tiñen

  en la negra sombra {]}
\vspace{2\baselineskip}

{[}solo s2{]}

A la corte vuelve

la confusa tropa

incierta qual suelen

discurrir las olas.
\vspace{2\baselineskip}

{[}solo t{]}

Quedé como muerto

aunque sea a mi costa,

que el que vive\footnote{\textsuperscript{}``muere'' en el original} amando,

pierde vida poca.
\vspace{2\baselineskip}

{[}a 4{]}

Descolgué de un sauze\footnote{\textsuperscript{}``mirto'' en el original},

mi ruda zamfoña,

y en dulces cadencias,

le canté esta copla

\vspace{4\baselineskip}
\textit{Estribillo:}
\vspace{1\baselineskip}

Más valéis vos Antona

que la corte toda.

La envidia os alabe

por humana Diosa.

Lograos, como fea,

matad como hermosa
\input{figures.tex}
\end{document}
