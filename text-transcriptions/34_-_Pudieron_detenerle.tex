\documentclass[titlepage,hidelinks]{article}
\usepackage[T1]{fontenc} 
\usepackage[utf8]{inputenc} 
\usepackage{multirow} 
\usepackage{tabularx} 
\usepackage{color} 
\usepackage{textcomp} 
\usepackage{tipa}
\usepackage{amsmath} 
\usepackage{amssymb} 
\usepackage{amsfonts} 
\usepackage{amsxtra} 
\usepackage{isomath} 
\usepackage{mathtools} 
\usepackage{txfonts} 
\usepackage{upgreek} 
\usepackage{enumerate} 
\usepackage{tensor} 
\usepackage{pifont} 
\usepackage{ulem} 
\usepackage{calc,xfrac}
\usepackage{arydshln} 
\usepackage[english]{babel}
\usepackage{subcaption}
\usepackage{pdfpages}
\usepackage{verse}
\usepackage{hyperref} 
\providecommand{\tightlist}{%
  \setlength{\itemsep}{0pt}\setlength{\parskip}{0pt}}


\DeclareCaptionFormat{custom}
{%
	\textbf{#3}
}
\captionsetup{format=custom,justification=raggedleft,font={Large}}
\hypersetup{
    colorlinks=true,
    linkcolor=cyan,
    urlcolor=blue,
  }


\date{}
\input{values.tex}

\begin{document}
\begin{titlepage} % Suppresses headers and footers on the title page
	\begin{center}
  \vspace*{\baselineskip} % White space at the top of the page
  \rule{\textwidth}{1.6pt}\vspace*{-\baselineskip}\vspace*{2pt} % Thick horizontal rule
  \rule{\textwidth}{0.4pt} % Thin horizontal rule
  \vspace{0.75\baselineskip} % Whitespace above the title
  \mytitle
  \vspace{0.75\baselineskip} % Whitespace below the title
  \rule{\textwidth}{0.4pt}\vspace*{-\baselineskip}\vspace{3.2pt} % Thin horizontal rule
  \rule{\textwidth}{1.6pt} % Thick horizontal rule
  \vspace{5\baselineskip} % Whitespace after the title block

{\Large Texto: \mytext 

Música: \mymusic}
\end{center}

\vspace{10\baselineskip} % Whitespace after the title block
\begin{flushleft}
Versión: \myversion

\textbf{Transcripción del texto:} \mytexttranscription

\textbf{Revisión del texto:} \mytextproofreading

\textbf{Validación del texto:} \mytextvalidation

\textbf{Transcripción de la música:} \mymusictranscription

\textbf{Revisión de la música:} \mymusicproofreading

\textbf{Validación de la música:} \mymusicvalidation

\textbf{Estudio poético:} \mypoeticstudy

\textbf{Estudio musical:} \mymusicalstudy 
\end{flushleft}
\end{titlepage}





\centering
\Large{

{[}solo s1{]}

Pudieron detenerle

suspiros, a mi amante

pero para que huyan

ellos le dan el aire.
\vspace{2\baselineskip}

{[}solo s2{]}

Porque son tales

que le soplan los vuelos

para ausentarse.
\vspace{2\baselineskip}

{[}solo t/ a 4{]}

antes no paren,

pues le hacen tan presente

cuando distante
\vspace{2\baselineskip}

{[}solo s1{]}

Del pecho los suspiros

y de los ojos los mares

hacen que amor o vuelva,

o también se embarque.
\vspace{2\baselineskip}

{[}solo s2{]}

Las olas y los aires

no son para que huyan

mas porque paren
\vspace{2\baselineskip}

{[}solo t/ a 4{]}

velas y naves

si llenan mis riquezas

glorias me hacen
\vspace{2\baselineskip}

{[}solo s1{]}

Del amor los suspiros

que de volcanes nacen

en llamas que se encienden,

porque no se acaben.
\vspace{2\baselineskip}

 {[}solo s2{]}

Suspiros que arden

a los que más consumen

más fríos hacen
\vspace{2\baselineskip}

{[}solo t/ a 4{]}

pueden constantes

hacer que amor que huye

al pecho abrase
\vspace{2\baselineskip}

{[}solo s1{]}

Al fin que estos suspiros

y estas finezas grandes

no pueden detenerle

los vuelos inefables.
\vspace{2\baselineskip}

{[}solo t/ a 4{]}

Pero más vale

que al amor que hoy se ausenta

solo lo acompañen

\vspace{4\baselineskip}
\textit{Estribillo:}
\vspace{1\baselineskip}

Ay, que me llevan suspiros, 

al aire, al bien de mis males

Ay Dios y qué crueldades

pues suspiros, debían pararle

a mi amante.
\input{figures.tex}
\end{document}
