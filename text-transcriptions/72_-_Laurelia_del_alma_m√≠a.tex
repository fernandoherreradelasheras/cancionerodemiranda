\centering
\Large{

Laurelia del alma mía,

tan discreta como hermosa

tu semblante es del cielo,

tienes la color de rosa.
\vspace{2\baselineskip}

Qué tiene tu parecer,

que solo\footnote{\textsuperscript{}``so'' en el original} con la vista mata

al que duerme sentido

pues solo el tuyo le falta.
\vspace{2\baselineskip}

No pienses flor mía

que soy amante ingrato

pues solo nombrarme tuyo

mis esperanzas rescato.
\vspace{2\baselineskip}

Toda la Sierra Morena

sabe lo mucho que te quiero

bien murmura pues te amo

con amor tan verdadero.
\vspace{2\baselineskip}

Las aves que van volando,

con los peces por la mar

y las fieras por la selva

huyendo de par en par.
\vspace{2\baselineskip}

Es de verme enmudecido,

por tus amores ingrata

vuelve pues Laurelia mía,

ya que tu amor me mata.
\vspace{2\baselineskip}

Duélete de tu pastor,

pues cansa de cantar

estos versos desiguales

que le sirven de llorar.

\vspace{4\baselineskip}
\textit{Estribillo:}
\vspace{1\baselineskip}

Canto si me miras,lloro si te escondes

ay Laurelia mía por qué no respondes.
