\subsection*{Criterios de edición musical}
\noindent La edición musical aquí incluida está adaptada a un perfil de músicos/intérpretes.

\noindent Todos los tonos se han transcrito a claves modernas. Aquellos escritos en claves altas (clave de Sol para los Tiple) se han transpuesto una cuarta justa hacia abajo adaptando la armadura. Tanto la infomación de las claves usadas en el original como la transposición si la hubiera se indican en los datos musicales de cada tono.

\noindent Las sílabas que se cantan sobre más de una nota, que en el original se indican mediante una ligadura o semimínimas agrupadas, se indican en la transcripción siempre con una ligadura entre las notas y un guión bajo continuo en el texto de cada una de las notas posteriores a la primera. 

\noindent Las alteraciones accidentales añadidas por semitonia subintelecta se han normalizado y se representan de manera normal. También se han añadido los accidentales necesarios cuando la nota mantiene la alteración del compás anterior. El resto de accidentales añadidos por motivos editoriales se indican en las notas críticas. Para una distinción explícita de alteraciones no includas en el manuscrito se puede consultar la edición adaptada a académicos/musicólogos y la versión web. 

\noindent La numeración de los compases de inicia desde 1 en cada sección.

\noindent Las expresiones dinámicas se han modernizado: pp por "brando".

\noindent La letra de todas las coplas se presenta debidamente alineadas con cada una de las voces. 

\noindent El compás ternario C3 se ha transcrito a 3/2 sin reducir los valores de las notas. Esto habrá de tenerse en cuenta de cara al tempo cuando en una misma sección se produzca un cambio a compás binario.

\noindent A fin de maximizar el espacio para la música en cada una de las páginas las notas a críticas a esta edición musical se incluyen en la siguiente sección indicando el compás y la voz a la que se aplican. 

