
El tono \textit{Aquella deidad del Tajo} un los romance en castellano de Frei António das Chagas (António da Fonseca Soares). El texto, que no permanece inédito, se conserva en, al menos dos fuentes manuscritas: \textit{Romances portugueses e castelhanos que compôs Frei António das Chagas antes de ser religioso}, P-Lant PT/TT/MSLIV/1726\footnote{\textsuperscript{}Disponible en https://digitarq.arquivos.pt/details?id=4248770} (p. 493-6) y otra compilación poética del autor titulada \textit{Obras en que se incluyen romances líricos castellanos de Antonio da Fonseca Soares}, conservada en la Biblioteca de Ajuda, P-La 49-III-79 (p. 47-51). Un tercer manuscrito, \textit{Poesías Varias, Portuguesas y Castellanas. Obras del insigne Fr. Antonio das Chagas} GB-Lbl Egerton 660 parece contener también una versión de este romance en el folio 192, según la descripción realizada por Pascual Gayangos en su primer volumen del \textit{Catalogue of the manuscritps in the Spanish language in the British Museum} de 1875. No ha sido posible consultar este manuscrito puesto que, a fecha de hoy (Enero 2025), la British Library aún no ha reestablecido los servicios de consulta y reproducción de manuscritos tras el ciberataque recibido en Octubre de 2023.
