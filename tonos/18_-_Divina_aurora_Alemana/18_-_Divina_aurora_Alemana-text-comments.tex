\noindent \textbf{1}: La transformación del poema original de Salazar y Torres, además del cambio de monarquía convierten los endecasílabos y heptasílabos del original en octosílabos más habituales en las composiciones vocales. Así la primera cuarteta se construye con versos de Salazar \textit{Divina aurora Alemana, a quien tributa reflejos} / \textit{El sol desde que nace} / \textit{Que aún el sol es vasallo de tu imperio}.\\
\textbf{5}: La segunda cuarteta toma solo el primer endecasílabo de la segunda estrofa de Salazar: \textit{A cuyos divinos rayos el Orbe es corto hemisferio}.\\
\textbf{8}, aún: se asume sinéresis para mantener la métrica.\\
\textbf{9}: La tercera cuarteta se construye a partir de un endecasílabo de cuarta estrofa salazariana: \textit{Que domine también en los afectos}.\\
\textbf{13}: Esta última estrofa, en donde se hace alusión al lazo de amor, es completamente nueva, difícilmente podría hacer referencia a ningún lazo de amor el texto de Salazar pues fue compuesto cuando Mariana de Austria ejercía de regente tras la muerte de Felipe IV.



