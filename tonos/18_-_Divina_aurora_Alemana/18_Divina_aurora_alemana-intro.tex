El texto del tono \textbf{Divina Aurora alemana} esta tomado, de maneral parcial, de un romance de Agustín Salazar y Torres que apareció publicado de forma póstuma en el primer volumen de la \textbf{Citara de Apolo}: \textit{A los felices años de la reina nuestra señora Doña Mariana de Austria} (p. 118). 


El tratado de Lisboa puso fin en 1668 a casi los casi 30 años de conflicto bélico entre Portugal y España. Sin embargo, las relaciones entre ambos países tardaron bastante tiempo en normalizarse. Por eso parece extraño que, en fechas comprendidas entre la creación del romance de Salazar y Torres, probablemente en 1670, para el 36º compleaños de Mariana de Austria\footnote{BELTRÁN DEL RÍO SOUS, Adriana: "Nuevo discurso de la vida y escritos de Agustín de Salazar y Torres", 2022. p 208} y los años posteriores a su publicación en 1681, en la Capilla Real portuguesa se lanzaran a cantar las glorias de la madre del rey de España.

Observando los cambios con respecto al romance original encontramos una posible explicación a su presencia en el Cancionero de Miranda: la reutilización del texto, quitando cualquier referencia a la monarquía hispánico, para cantar en Lisboa a otra monarca alemana. Esta situación se produciría a partir de 1687 cuando María Sofía Isabel del Palatinado-Neoburgo contrajo matrimonio con Pedro. Los versos añadidos \textit{Viva el lazo indisoluble // de amor en tu dulce empeño} parecen indicar que además de cambiar la reina, se cambió el cumpleaños por la boda. El rizo se terminaría de rizar en 1689 cuando, a las dos deidaes alemanas de nuestros tonos, Mariana y María Sofía, se les una una tercera, también Mariana: Mariana del Palatinado-Neoburgo, segunda esposa de Carlos II y hermana de María Sofía. 

Otro importante testimonio de la circualación de tonos humanos dirigidos a los monarcas ibéricos lo encontramos en un volumen fáctico que reune diversa documentación y correspondencia de la reina María Sofía. Este manuscrito consevado en Paris, F-Pn Portugais 32, abarca todo el periodo portugués de María Sofía hasta su muerte en 1699. Entre la numerosa correspondencia familiar no faltan los intercambios con su hermana la reina de España, a los que acompañarían seguramente el romance que con el que se celebró uno de los cumpleaños de Mariana de Neorbugo\footnote{La ficha bibliográfica de la BNF asocia por error este romance la reina madre Mariana de Austria. El índice del manuscrito solo indica \textit{Cançao aos annos da Rª de Hespanha} y el contenido del poema, que insiste una vez más en su fertinilidad para darle un heredero al rey, no deja duda alguna}


Como curiosidad incluimos el texto de este romance intercambiado entree las reinas.



\begin{multicols}{3}
\fontsize{9.2pt}{10pt}\selectfont
\raggedcolumns

\settowidth{\versewidth}{xxxxxxxxxxxxxxxxxxxxxxxxxxxxxxxxx}
\begin{verse}[\versewidth]
\begin{altverse}
Oy, grande reina, tus años \\
no solo España celebra \\
mas todo el mundo, pues todo \\
rinde a tu pie su cabeça \\!
Tus años festiva Hespaña \\
de justicia los festeja \\
y las mas partes del mundo \\
te los applaude por deuda \\!
Con razon porque eres Sol \\
que ilustra toda su esfera \\
y a donde llegan tus luces \\
todos los bienes le llegan \\!
Germania appluaude tus años \\
porque tu Oriente fue en ella \\
y de haber sido tu cuna \\
tiene gloriosa soberbia \\!
Celebra Italia tus años \\
porque como en ella Reunas \\
de ser tu tu vasalla tiene \\
de mas otras indulgencias \\!
Asia tus años estima \\
porque en el oriente impresa \\
y desde las Philipinas \\
te tributan obediencias \\!
America los aplaude \\
con plata, oro y con perlas \\
porque perlas, oro y plata \\
se ??? todo en ti misma \\!
Africa tambien adora \\
tus años y desde Cueta \\
la africana media luna \\
quiere verte luna llena \\!
Mas Porgutal mas que todos \\
tus años estima y precia \\
pues desde Madrid tus luces \\
en Lisboa reverberan \\!
De portugual e eminencia \\
de Lusitania la reina \\
de su escudo sino quinas \\
de su fruto sino estrellas \\!
Festejan mejor que todos \\
tus alegres primaveras \\
pidiendo a Dios que tus flores \\
bre?te frutos sean. \\!
Que en lo fecundo a tu Hermana \\
como en lo mas te parezcas \\
para que asi multiplques \\
años en tus descendencias \\!
Que hagas Padre al que es esposo \\
y el a ti madre y te viera \\
con tantos hijos como años \\
tienes, y todos desean. \\!
Para que del grande Carlos \\
por ti la edad sea eterna \\
pues eterniza la vida \\
la sucesión que se deja \\!
Vive Reina soberana \\
y tu edad feliz numera \\
por las estrellas del Cielo \\
o las perlas cibreas? \\!
Sean tus años iguales \\
a tos meritos y prendas \\
y por ti la edad dorada \\
para toda España venga \\!
\end{altverse}
\poemlines{0}
\end{verse}

\end{multicols}

\vfill
