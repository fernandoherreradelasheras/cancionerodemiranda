El texto del tono \textbf{Divina Aurora alemana} esta tomado, de maneral parcial, de un romance de Agustín Salazar y Torres que apareció publicado de forma póstuma en el primer volumen de la \textbf{Citara de Apolo}: \textit{A los felices años de la reina nuestra señora Doña Mariana de Austria} (p. 118). 


El tratado de Lisboa puso fin en 1668 a casi los casi 30 años de conflicto bélico entre Portugal y España. Sin embargo, las relaciones entre ambos países tardaron bastante tiempo en normalizarse. Por eso parece extraño que, en fechas comprendidas entre la creación del romance de Salazar y Torres, probablemente en 1670, para el 36º compleaños de Mariana de Austria\footnote{BELTRÁN DEL RÍO SOUS, Adriana: "Nuevo discurso de la vida y escritos de Agustín de Salazar y Torres", 2022. p 208} y los años posteriores a su publicación en 1681, en la Capilla Real portuguesa se lanzaran a cantar las glorias de la madre del rey de España.

Observando los cambios realizados al romance original podríamos explicar su presencia en el Cancionero de Miranda como la reutilización del texto cuando hubo en Lisboa también una monarca alemana a la que cantar: María Sofía Isabel del Palatinado-Neoburgo, esposa de Pedro II desde 1687. Así desaparecen los versos que nombran a Carlos o la unión entre el águila y el león. Así mismo, los versos añadidos \textit{Viva el lazo indisoluble // de amor en tu dulce empeño} parecen indicar que además de cambiar la reina, se cambió el cumpleaños por la boda.

