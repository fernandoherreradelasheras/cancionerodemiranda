El tono a 4 \textbf{Airecillos mansos} aparece como anónimo tanto en el Cancionero de Miranda como en el Cancionero poético-musical hispano de Lisboa. El tercero de los testimonio conservados, en la colección de villancicos de la Biblioteca Pública de Évora, es el que nos informa de su autor mediante la anotación del apellido \textit{Marques} en la parte del guión y luego su nombre completo en la carpetilla (de época posterior): \textit{Ant\textsuperscript{$\infty$} Marques Lesbio}. Fue cantado en la Capilla Real de Lisboa en la Navidad de 1681, cuyo pliego impreso lo recoge no como villancico sino como ``MISSA``. 

La parte del libro de tenor de \textit{Airecillos mansos} en el Cancionero de Miranda no lleva ninguna indicación específica pero su equivalente en el Cancionero poético-musical hispano de Lisboa está anotada como Alto. La voz perdida en ambos cancioneros se corresponde con el alto 1º de la versión de Évora. La parte vocal es casi idéntica en las tres versiones, con un par de notas diferentes y alteraciones que aparecen escritas en una versión mientras que en otra se asumen como semitonia subintelecta. En la parte de acompañamiento, la copia de Évora y la del Cancionero de Miranda son iguales, mientras que la del Cancionero poético-musical hispano de Lisboa presenta alguna diferencias (sobre todo cambios de octava de algunos grupos de notas pero también en cuanto al reparto de algunas duraciones).

En cuanto al texto, la versión de Évora incluye solo las tres primeras coplas de las cinco de los otros dos testimonios musicales que a su vez omiten la última copla incluida en el pliego. 

Varios versos del texto los podemos trazar a lo largo de todo el siglo XVII en diversas encarnaciones a lo humano y a lo divino. Así, los versos del estribillo:

\begingroup
\centering
\itshape
\begin{verse}
paso, quedito,\\
que duerme mi niño\\
\end{verse}
\endgroup

recuerdan aquellos que canten los músicos en la tercera jornada de \textit{El ruiseñor de Sevilla} de Lope de Vega, escrita entre 1603 y 1606:

\begingroup
\centering
\itshape
\begin{verse}
quedito, pasito, amor,\\
no espantéis al ruiseñor\\
\end{verse}
\endgroup

En un villancico cantado en la Capilla Real de Madrid en 1651\footnote{http://bdh-rd.bne.es/viewer.vm?id=0000060876} encontramos tanto el \textit{paso quedito} como los \textit{airecillos} y \textit{no recuerde}: 

\begingroup
\centering
\itshape
\begin{verse}
Pasitico, airecillos, que se duerme el Sol:\\
queditico, avecillas, no recuerde amor\\
suspended la voz\\
no le recordéis\\
ni le desveléis\\
ventecillos no\\
pajarillos no\\
que harto desveladico me le tengo yo\\
\end{verse}
\endgroup

Es probable que este villancico de 1651 fuera compuesto por Carlo Patiño, pues esa autoría consta en una readaptación para el Santísimo Sacramento conservada en Canet de Mar (Au 788)\footnote{Editada por Mariano Lambea y Lola Josa en https://digital.csic.es/handle/10261/156526}. Encontramos el mismo texto en un villancico cantado en Córdoba en la Navidad de 1665. 

De nuevo en la vertiente profana, en 1662 se estrena la comedia de Calderón Ni amor se libra de amor, donde se canta:

\begingroup
\centering
\itshape
\begin{verse}
Quedito, pasito,\\
que duerme mi dueño,\\
quedito, pasito,\\
que duerme mi amor.\\
\end{verse}
\endgroup

La cancioncilla de la comedia de Calderón fue puesta en música por Juan Hidalgo (conservada en el manuscrito Gayangos-Barbieri). Este tono de Hidalgo sería reutilizado por Miguel Gómez Camargo como villancico cantado en la Catedral de Valladolid la Navidad de 1663. El mismo Camargo reelaboraría texto y música al menos tres veces más: un villancico a 8 para el Corpus de 1666, otro a 5 para el Corpus de 1677 y otro a 8 para el Corpues de 1683.

Valga como último testimonio de la enorme popularidad de estos versos y motivos citar los versos del villancico que se cantó inmediatamente antes de \textit{Airecillos mansos} aquella Navidad de 1681 en el Capilla Real de Pedro II:

\begingroup
\centering
\itshape
\begin{verse}
Quedo, paso,
que el sol se duerme,\\
avecillas vuestros gorjeos,\\
fuentecillas, vuestras corrientes,\\
no le despierten.\\
\end{verse}
\endgroup
