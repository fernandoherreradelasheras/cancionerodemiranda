\noindent \textbf{1} ``recordar``: Metafóricamente vale despertar al que está dormido. (Diccionario de Autoridades)\\
\textbf{10}: ``pareçe`` en nuestro manuscrito, de igual grafía en Évora, ``parese`` en el CPMHL y ``paresse` en el pliego impreso. Consideramos que ``párese`` en la opción que mejor encaja semánticamente (ordenar a un viento frío que se detenga porque el niño está desnudo), en cuanto al ritmo (todas las coplas empiezan en sílaba tónica) y como parte de una figura de repetición en la que las coplas segunda y cuarta también comienzan con un imperativo.\\
\textbf{13}: El ``no'' falta en superius primus.\\
\textbf{15}: Interpretamos párese de manera análoga al verso 10.\\
\textbf{16}: ``despierto`` en nuestro manuscrito, ``despierta``en el CPMHL y en el pliego impreso y ``despierte`` en la versión de Évora. Tomamos esta última.\\
\textbf{20}: mantenemos el original ``arboledos`` que aperece igual en el CPMHL y en el pliego impreso.\\
\textbf{30}: esta copla la tomamos del pliego impreso. 
