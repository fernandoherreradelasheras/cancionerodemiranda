\subsection*{Datos musicales}
\noindent \textbf{Orgánico}: 4 voces (Tiple 1º, Tiple 2º, Alto 1º, Alto 2º) y Acompañamiento (Guión)\\
\textbf{Claves bajas}: Tiple 1º (Do en 1ª). Tiple 2º (Do en 1ª). Alto 1º (Do en 3ª). Alto 2º (Do en 3ª). Acompañamiento (Fa en 4ª)\\
\textbf{Armadura}: Un bemol\\
\textbf{Transcripción}: a claves modernas sin transposición

\subsection*{Notas a la edición musical}
Para la voz del Alto 1º, perdida en el Cancionero de Miranda, hemos tomado la versión de Évora.

\noindent \textbf{Compás 2, alto 2º}: el fa no aparece sostenido ni nuestro manuscrito ni en al CPMHL pero sí en la partitura de Évora, del mismo modo que lo hace para el Alto 1º en el primer compás. Asumimos pues esta alteración como semitonía subintelecta en la repetición del motivo en las otras voces.\\
\textbf{Compás 42, alto 2º}: adelantamos el becuadro al primer si, tal y como sucede en el guión. Dejamos el segundo si con bemol siguiendo la partitura de Évora.\\
\textbf{Compás 44, alto 2º}: re con sostenido en el manuscrito que resulta incompatible con el re natural del guión y que tampoco aparece en el CPMHL ni en Évora.\\
\textbf{Compás 49, tiple 2º y 51 alto 2º}: La partitura de Évora mantiene en ambos casos el primer mi natural y aplica el bemol al segundo, que da lugar a un cromatismo interesante.

