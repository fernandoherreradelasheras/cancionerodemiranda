\noindent \textbf{3}: ``si dan vida tus ojuelos`` en la versión impresa del romance.\\
\textbf{5}: ``más ya lo entiendo`` en la versión impresa del romance.\\ 
\textbf{6}: ``no es difícil de acertarte`` en la versión impresa del romance.\\ 
\textbf{9}: Esta cuarteta no aparece en el tono. La recuperamos para dar mejor sentido a la siguiente.\\
\textbf{13}: ``tus envidias`` en manuscrito, que al no incluir la cuarteta anterior perdía el contexto para ``sus envidias`` (de las vecinas)\\
\textbf{14}: ``hermosas zagalas`` en manuscrito, que de nuevo, no tenía el contexto (los ojos) al que hace referencia el ``hermosos y graves`` de la versión impresa. Hemos optado por mantener la versión de romance publicado puesto que la del tono rompía la rima asonante de los impares.\\
\textbf{15}: Es interesante que, tanto la parte de tiple 2º como la tenor, escriben ``pero parece que mataran`` mientras que la del tiple 1º mantiene el ``juré que mataban`` del texto original. Parece que sería el compositor o el copista el que corriera los versos que no tenían sentido al no incluir la cuarteta anterior.\\
\textbf{17}: ``el Alba de San Juan`` en la versión impresa.\\
\textbf{21}: Esta cuarteta y la anterior tienen el orden intercambiado en el tono. Tomamos el orden de la versión impresa para mantener el discurso narrativo entre las dos últimas cuartetas. \\
\textbf{22}: ``romances`` en la versión impresa.\\
\textbf{23}: ``vienen al valle`` en la versión impresa.\\
\textbf{26}: ``pudieres`` en la versión impresa.
