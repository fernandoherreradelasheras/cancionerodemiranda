\textit{Del silencio de este valle} es uno los tonos del Cancionero de
Miranda a los que, por estar escrito para dos tiples y bajo, no le
afecta la desaparición del libro de alto y está completo sin necesidad
de ninguna reconstrucción.

El texto es es un romance con estribillo de \textbf{Francisco de Borja y
Aragón} incluido con el número CCXXXIX en \textit{Las Obras en verso de
Don Francisco de Borja, Principe de Esquilache} publicadas en Amberes
1663. Del las nueve cuartetas que tiene el romance, el tono incluye las
cinco primeras.

Existe otro tono con este mismo romance (coplas primera, segunda, cuarta
y séptima) compuesto por el \textbf{Marqués de Cábrega}. Se trata de un
solo humano y se conservan tres copias diferentes:

\begin{itemize}
\tightlist
\item
  Manuscrito E-Bbc M 1637-II/13 de la Biblioteca de Catalunya procedente
  de la Colegiata de Verdú\footnote{Esta fuente fue utilizada por Lola
    Josa y Mariano Lambea para la edición del tono incluída en ``Todo es
    amor. Manojuelo Poético-Musical de Barcelona'', 2013. El texto y la
    partitura de esta edición están disponibles libremente en
    \url{https://digital.csic.es/handle/10261/138298/}}
\item
  Manuscrito E-Bbc M 738/5 de la Biblioteca de Catalunya procedente de
  la colección de Joan Carreras i Dagas
\item
  Folio 112 del manuscrito Gayangos-Barbieri E-Mn MSS/13622 de la
  Biblioteca Nacional de España\footnote{Copia digital disponible en
    \url{http://bdh-rd.bne.es/viewer.vm?id=0000129924}}
\end{itemize}

Una última diferencia entre ambos tonos es la posición del estribillo:
tras las coplas en el tono a 3 del Cancionero de Miranda y al comienzo
en el Solo de Cábrega. La versión publicada del romance repite el estribillo
cada tres cuartetas.
