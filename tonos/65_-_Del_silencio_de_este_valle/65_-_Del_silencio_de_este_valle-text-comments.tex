\noindent \textbf{2}: En el manuscrito aparece ``no vuelve a ver''. En esa forma faltaría el objeto del verbo y no tendría mucho sentido. Corregimos con el texto publicado del romance.\\
\textbf{2}: Mantenemos la variante ``Amariles'' del manuscrito frente a ``Amarilis'' de la versión publicada en las Obras en verso de Don Francisco de Borja que parece un error pues rompe la rima.\\
\textbf{3}: En el manuscrito ``en tristezas se falta''.\\
\textbf{18}: En el manuscrito ``ofendido es humilde''.
