\noindent La repetición del primer verso indicada mediante un símbolo de repetición en las partes de las voces se ha transcrito duplicando los compases puesto que el acompañamiento es ligeramente diferente.


\noindent El guión incluye un rojão, esto es, un preludio (o interludio) instrumental equivalente portugués al pasacalle. Por tanto, además de tocarse entre cada repetición de las coplas se puede tocar también antes de la primera\footnote{Para más información sobre este tipo de prelucios véase Rojão: um prelúdio Português no séc. XVIII: manuscrito musical P-CUG MM 97, Biblioteca Geral da Universidade de Coimbra}

TODO: Para el signo de repetición tras el rojão, confirmar si queda mejor colocar una barra de repetición tras la semibreve dejando luego un silencio y la barra de final de compás o directamente omitir el último silencio suelto.
