
\subsection*{Datos musicales}
\noindent \textbf{Orgánico}: 4 voces (Tiple 1º, Tiple 2º, [Alto], Tenor) y acompañamiento (guion)\\
\textbf{Claves altas}: Tiple 1º (Sol en 2ª). Tiple 2º (Sol en 2ª). Tenor (Do en 3ª). Acompañamiento (Do en 4ª).\\
\textbf{Armadura}: un bemol\\
\textbf{Transcripción}: a claves modernas transpuestas una cuarta hacia abajo, armadura sin alteraciones

\subsection*{Notas a la edición musical}

\noindent La repetición del primer verso indicada en el manuscrito mediante barras de repetición en las partes de las voces se ha transcrito expandiendo los compases puesto que la música del acompañamiento es ligeramente diferente.


\noindent El guión incluye un rojão, esto es interludio instrumental equivalente portugués al pasacalle español, para ser tocador entre una copla y otra\footnote{Para más información sobre este tipo de prelucios véase Rojão: um prelúdio Português no séc. XVIII: manuscrito musical P-CUG MM 97, Biblioteca Geral da Universidade de Coimbra}.

