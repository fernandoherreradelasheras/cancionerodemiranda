
El tono \textit{A dónde corres, arroyo} pone en música un romance en castellano de Antonio Barbosa Bacelar que gozó de amplia difusión al ser incluido en el tomo segundo de A Fénix Renascida\footnote{Serie en cinco partes compilada por Matias Pereira da Silva entre 1716 y 1728. Copias digitales disponibles en https://purl.pt/261/} (p.167). Del largo romance de Barbosa Bacelar el tono incluye las cuartetas 1ª, 2ª, 3ª, 13ª, 17ª y 18ª y no toma el estrbillo.


