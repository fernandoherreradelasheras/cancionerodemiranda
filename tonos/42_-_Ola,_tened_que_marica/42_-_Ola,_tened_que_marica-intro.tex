
Dos de cumbres del barroco portugués se funden en el tono \textbf{Hola, tened que Marica}: la música de Antonio Marques Lesbio y la poesía de don Francisco Manuel de Melo.

Este tono del Cancionero de Miranda constituye una agradable sorpresa. Entre los numerosos poemas melodinos para los que tenemos constancia de que fueron puestos en música no se encontraba este romance de La Citara de Erato publicado como número 54º y el sobretítulo "D. caida".

De las 18 coplas del romance original el tono copia las 3 primeras y el estribillo. 

