El tono \textbf{Ya no quiero navegar}, compuesto por textbf{Joao Lorenzo Rebello}, comparte fuente poética con el villancico homónimo a 5 voces de la colección de la Biblioteca Pública de Évora\footnote{P-EVc Ms. CLI/1-5 d nº15}. Pese a tratarse de una obra a lo humano y otra a lo divino, el texto del estribillo es completamente igual. Las tres coplas del uno del villancico, sin embargo, parecen no guardar ninguna relación con el tono del Cancionero de Miranda: \textit{No más me quiero engolfar}, \textit{Por aquel mar de María} y \textit{Del mar del mundo corsario}. 

Que el tono presente una única copla es, desde luego, una rareza. El espacio reservado para los versos en las tres partes del manuscrito parece ser indicio de que existían dichas coplas y por algún motivo no se copiaron. 

