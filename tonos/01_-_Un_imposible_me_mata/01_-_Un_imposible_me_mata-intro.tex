El Cancionero de Miranda se abre con un bloque de 22 tonos breves sin estribillo. En ninguno de estos tonos aparece indicación acerca del compositor y tampoco se ha podido localizar ninguno por concordancias. Del mismo modo que los tonos de este bloque son homogéneos en cuanto a la forma, también parecen compartir un rasgo sobre su origen poético. De los poetas indentificados, abundan los textos de poetas portugueses: varios de Frei António das Chaga, António Barbosa Bacelar y Miguel de Barrios, que pese a ser Cordobés su familiar era de origen portugués.

No conocemos la autoría del romance \textbf{\textit{Un imposible me mata}}, pero una de sus coplas, \textit{Triste el corazón se queja}, aparece en la \textit{Colección de las mejores coplas, de seguidillas, tiranas y polos que se han compuesto para cantar a la guitarra} recopilada y \textit{Juan Antonio de Iza Zamácola} y publicada por primera vez en 1799.
