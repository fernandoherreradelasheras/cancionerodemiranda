El tono humano \textbf{Qué valentones que campan} constituye un buen ejemplo de la hiperbolización del detalle en los poemas exaltación de la belleza del siglo XVII\footnote{LAMBEA, Mariano y JOSA, Lola: ``Señas de una belleza superior`` o las representaciones del cuerpo en el tono humano barroco. 2009 http://hdl.handle.net/10261/22355}. En este caso, el poeta se centra en los pies de Jacinta, pequeños como corresponde al canon de belleza femenino de la época. Y del mismo modo que los ojos y las manos, los pies se nos presentan como armas implacables. Resulta interesante el uso reiterado de términos que nos remiten al mundo de la jácara como pueden ser valentones, cuchilladas, desgarrado.

El texto poético concuerda parcialmente con el que aparece recopilado en un manuscrito conservado en la Biblioteca Ajuda, P-La 49-III-52 (p. 275). La primera cuarteta es igual en ambas versiones. La segunda cuarteta del manuscrito de Ajuda mezlca elementos de la cuarta y de la tercera del tono. La tercera y última cuarteta del manuscrito de Ajuda introduce dos versos nuevos (\textit{Y aunque lóbregos estén} / \textit{muy claramente divisan}) que se completan con la referencia a las medias de la tercera cuarteta del tono. No tiene estribillo la versión de Ajuda.
