que sentillaban\footnote{\textsuperscript{}Adaptarlo a centelleaban rompería la métrica del verso} tiernas.
