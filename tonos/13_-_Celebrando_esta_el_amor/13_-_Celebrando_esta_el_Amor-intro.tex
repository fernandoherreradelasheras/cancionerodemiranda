El breve tono \textit{Celebrando está el amor}, aún configurado en una sola sección, repite el mismo verso \textit{alamar no, sino alarma} como pie de cierre de cada estrofa.

La primera cuarteta de nuestro tono humano coincide en gran parte con la primera de un romance de \textbf{Antonio Hurtado de Mendoza} copiado en el folio 57r del manuscrito E-Mp II/2802:

\begin{verse}
Celebrando está el amor \\
las travesuras de Juana, \\
guerras que a todos les toca\\
al amor no, sino al alma.\\
\end{verse}

Resulta interesante la transformación entre \textit{al amor no, sino al alma.} y \textit{alarma, no sino alarma} por la que el amor y el alma se desplazan hasta alarmas y armas. 

Sí comparte el nombre de Anarda, además de la idea la belleza como arma que embite y avasalla, con uno de los tonos más bellos del del \textbf{Libro de Tonos humanos}:  \textit{A desafiar las flores}

\begin{verse}
A desafiar las flores, \\
¡qué linda Anarda salió!; \\
bien puede dársele el vítor, \\
que es cobarde toda flor.
\end{verse}

\begin{verse}
¡Al arma, al arma,\\
plantas y flores,\\
que las armas de Anarda\\
el campo reconoce,\\
y ya embisten sus soles!
\end{verse}

