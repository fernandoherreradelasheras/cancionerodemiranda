El tono \textbf{Su hielo esgrime la noche} posee una estructura atípica, con dos grupos de coplas y dos estribillos, que hizo que en el artículo en el que se daba cuenta del cancionero se identificara como dos tonos diferentes. Su estructura es:

\begin{itemize}
\item Coplas: \textit{Su hielo esgrime la noche} // \textit{Con la más villana sombra} // \textit{Contra amor que arroja perlas} // \textit{Dos contrarios en un sujeto}
	\item Estribillo: \textit{Despertad zagalejos}
	\item Coplas (a solo y a cuatro): \textit{Helado sol en llamas} // \textit{En brazos del alba} // \textit{Mucho dices callando} // \textit{Si mucho te explicas} // \textit{Hoy es de tu fineza} // \textit{Que suelen finezas} 
	\item Estribillo \textit{Pajarillo que cantas amante}
\end{itemize}

Los dos grupos de coplas de este tono aparecen un villancico, también doble, que se cantó en la Iglesia del Pilar de Zaragoza en 1668 y que se recopilan en la \textbf{\textbf{Lyra poética}} de Vicente Sánchez:

\begin{itemize}
	\item [sin encabezado] \textit{Su hielo esgrime la noche} // \textit{Con la más villana sombra} // \textit{Contra amor que arroja perlas} // \textit{De hombre y Dios, hacer compuesto}
	\item Estribillo: \textit{Tírele, tírele hielos}
	\item Coplas: \textit{Helado sol en llamas} // \textit{En brazos del alba} // \textit{Mucho dices callando} // \textit{Si mucho te explicas} // \textit{Hoy es de tu fineza} // \textit{Que suelen finezas} // \textit{Rendidos tres monarcas} //  \textit{Pues dones que ofrecen}
	\item Estribillo \textit{Desnudo amor, dulce flechero}
\end{itemize}

Al año siguiente, la letra del villancico de Sánchez se canta en las celebraciones de la Navidad de la catedral de Córdoba, manteniendo los dos grupos de coplas y el primer estribillo pero cambiando el segundo:

\begin{itemize}
	\item Coplas: \textit{Su hielo esgrime la noche} // \textit{Con la más villana sombra} // \textit{Contra amor que arroja perlas} // \textit{De hombre y Dios, hacer compuesto}
	\item Estribillo: \textit{Tírele, tírele hielos}
	\item Coplas: \textit{Helado sol en llamas} // \textit{En brazos del alba} // \textit{Mucho dices callando} // \textit{Si mucho te explicas} // \textit{Hoy es de tu fineza} // \textit{Que suelen finezas}
	\item Estribillo \textit{A Belén zagales que ha nacido el Sol}
\end{itemize}


Unos años más tarde, Benito Bello de Torices volvería a utilizar parte del texto de Vicente Sánchez para un villancico que conocemos en la recopilación que realizada por Gerónimo Vermell. Toma Torices la primera de las coplas del primer grupo a modo de sección de introducción, y las coplas del segundo grupo (aunque Vermell solo copia la letra de la primera). Lo curioso es el único estribillo que elige Torices: el mismo que el que aparece en el Cancionero de Miranda y que no aparecía en los villancicos anteriores:

\begin{itemize}
	\item Introducción: \textit{Su hielo esgrime la noche} 
	\item Estribillo: \textit{Despertad zagalejos}
	\item Coplas: \textit{Helado sol en llamas} 
\end{itemize}


El estribillo \textit{Despertad zagalejos} que comparten el tono del Cancionero de Miranda y el villancico de Torices lo encontramos no muy lejos de Zaragoza: es el estribillo de en los villancicos cantados en la catedral de Barbastro en la Navidad de 1677.

Completar el rompecabezas del tono del Cancionero de Miranda nos lleva de nuevo al mundo de \textit{lo humano} y terreno teatral. El estribillo final, \textit{pajarillo que cantas amante}:

\begingroup
\centering
\itshape
\setlength{\vrightskip}{-3em}
\begin{verse}
Pajarillo que cantas amante,\\
¡calla, no cantes! ¡No cantes detente!\\
Llorar es mejor\\
que no cabe en tus voces\\
el gozo de oír llorar\\
Suspende, suspende el canto\\
pues tiene Amor en el llanto\\
más cariñosa canción.\\
\end{verse}
\endgroup


es una reelaboración con muy pocos cambios del estribillo de la canción que canta Orfeo en la comedia \textit{Eurídice y Orfeo} de \textbf{Antonio de Solís} estrenada en 1643 y respuesta en Madrid 1655:

\begingroup
\centering
\itshape
\setlength{\vrightskip}{-3em}
\begin{verse}
Pajarillo que cantas ausente,\\
¡calla, no cantes! ¡No cantes, detente!\\
Llorar es mejor\\
que no caben ausencia, dulzura y amor.\\
Llorar, llorar es mejor.\\
Suspende, suspende el canto\\
pues puso Amor en el llanto\\
la música del dolor.\\
\end{verse}
\endgroup


Cristobal Galán compuso Un tono humano a solo con el texto completo, incluidas las coplas, de esta canción de Solis\footnote{US-NYhsa 380/824a/38,1}. El estribillo de Galán aparecerá en 1694 insertado en la ópera Bassiano de Alessandro Scarlatti\footnote{NESTOLA, Barbara: Musica e stile francese nell’opera italiana tra Venezia, Roma e Napoli (1680–1715). Publicado en Analecta musicologica 52: Europäische Musiker in Venedig, Rom und Neapel (1650–1750)}

Aún cononciendo el origen de todas las partes de este complejo tono, solo podemos conjeturar acerca de los procesos de reelaboración que dieron lugar al texto final. Por un lado, la naturaleza \textit{humana} de nuestro tono explica las discrepancias con el villancico de Vicente Sánchez. Así del primer grupo de coplas, el texto de la última de Sánchez:

\begingroup
\centering
\itshape
\setlength{\vrightskip}{-3em}
\begin{verse}
De Hombre, y Dios hacer compuesto\\
amor quiere en dulce uníon,\\
que lo eleva, aquí del hombre \\
que lo humilla, aquí de Dios.\\
\end{verse}
\endgroup


se cambia sutilmente para incidir en la figura de Cupido mientras quita peso al contraste hombre/Dios:

\begingroup
\centering
\itshape
\setlength{\vrightskip}{-3em}
\begin{verse}
Dos contrarios en un sujeto,\\
hoy quiere unir el amor\\
que lo alaba, aquí de un ciego\\
que lo humilla, aquí de un Dios.\\
\end{verse}
\endgroup

Y del segundo grupo, directamente no se incluyen las dos últimas que hace referencia a los Reyes Magos. Por otro lado, que nuestro tono y el villancico de Torices complementen las coplas de Sánchez con un mismo estribillo, algo improbable que ocurriera de manera independiente, sugiere la existencia de un texto humano anterior, que incluyera el estribillo \textit{Despertad zagalejos}, del que habrían derivado los textos del Cancionero de Miranda y del villancico de Torices y, por otro lado, se habrían divinizado en los villancicos cantados en Zaragoza y Cordoba. Este hipotético texto podría incluir una variación de una hipotética canción popular que incluría Solis en su comedia, aunque cualquier otra combinación sería posible. En cualquier caso, resulta interesante la coincidencia entre el \textit{no cantéis,
llorar es mejor} del primer estribillo con el \textit{no cantes detente, llorar es mejor} del segundo.


\todo{Añadir los enlaces a la edición del villancico de Torices por Lambea y a los pliegos de villancicos}
