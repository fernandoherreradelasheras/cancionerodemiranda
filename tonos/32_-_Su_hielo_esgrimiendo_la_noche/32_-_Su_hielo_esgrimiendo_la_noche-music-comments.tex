\subsection*{Datos musicales}
\noindent \textbf{Orgánico}: 4 voces (Tiple 1º, Tiple 2º, Alto, Tenor)\\
\textbf{Claves altas}: Tiple 1º (Sol en 2ª). Tiple 2º (Sol en 2ª). Tenor 2º (Do en 3ª).\\
\textbf{Armadura}: Un bemol\\
\textbf{Transcripción}: a claves modernas, transpuesta una cuarta hacia abajo. Armadura sin alteraciones.

\subsection*{Notas a la edición musical}

\noindent \textbf{compás 15, Tenor}: semibreve y silencio de mínima. Corregimos a semibreve con puntillo para igualarlo a la otra voz\\
\textbf{compás 28, Tiple 1º}: -> edit el sostenido\\
\textbf{compás 35 Tiple 1º}: la semibreve va sin puntillo, añadido para igualarlo con las otras dos voces\\
\textbf{compás 37, Tiple 1º}: -> edit el primer sostenido\\
\textbf{compás 38, Tiple 1º}: -> edit los sostenido\\
\textbf{compás 52, Tiple 1º}: > la ultima nota es semiminima pero interpretamos que la intencion era ligar una minima\\
\textbf{compás 77,, Tiple 2º}:  -> edit el mib\\
\textbf{compás 104, Tiple 1º}: edit -> adelantamos el sostenido\\
\textbf{compás 137, Tiple 1º}: solo una semiminima en el ms, duplicamos \\
\textbf{compás 164, Tiple 2º}: dos minimias en el ms. Hacemos la segunda semibreve para que encajen las duraciones\\
\textbf{compás 169, Tiple 2º}: minima con puntillo y semiminima que dejamos en dos mínimas en consonancia con las otras voces\\
\textbf{compás 151, Tenor}: tres minimas en el ms, hacemos la ultima semibreve por encaje\\

