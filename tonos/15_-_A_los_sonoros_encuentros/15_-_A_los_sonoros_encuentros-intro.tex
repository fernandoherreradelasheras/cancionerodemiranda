El tono \textbf{A los sonoros encuentros} es una pieza de circustancia típica en la que se le da la bienvenida a una \textit{deidad alemana}. Si fuera de origen portugués, esta deidad podría ser María Sofía Isabel del Palatinado-Neoburgo, que, como segunda esposa de Pedro II, fue reina consorte de Portugal entre 1687 y 1699. No hay que descartar, sin embargo, que pudiera tener su origen en la monarquía española. Aunque parezca extraño el uso de un texto ``rival`` el entorno de la capilla real portuguesa, existen ejemplos en el Cancionero de Miranda como el tono nº18, \textit{Divina aurora alemana} cuyo texto dedicó Salazar y Torres a \textit{los años de la reyna nuestra señora Mariana de Austria}. Aunque la compilación la realizara un cantor de la capilla real, que su destino fuera una institución externa habría permitido incluir textos que no eran aceptables para consumo de la propia capilla.
