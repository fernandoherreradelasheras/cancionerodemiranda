El largo tono \textbf{Más valéis vos, Antona}, sobre el texto de unas Endechas de Gabriel Bocángel, es una de las piezas formalmente más complejas del Cancionero de Miranda. Sus quince coplas, cuartetas heaxilábicas, se reparten de manera irregular entre cinco configuraciones: las dos primeras y la última copla se canta a 4 voces y entres ellas, se van alternando los cuatro solistas según el orden Tiple I (3ª, 7ª y 11ª), Alto (4ª, 8ª y 12º), Tiple II (5ª, 9ª y 13ª) y Tenor (6ª, 10º y 14ª). El largo estribillo (103 compases a 3/2) aparece al principio y sorprende que dada su extensión, se dejara fuera una sola copla del poema original. No sabemos si se dejó fuera \textit{Toda la zagala} o \textit{Expiraba el día} poque la omisión de saltarse una de esas dos coplas y tomar la otra le corresponde a la parte perdida del Alto. La música de los solos es prácticamente idéntica para todas las partes aunque el tiple I y el Tenor arrancan en el ``dar`` del compás con una nota extra frente las dos otrs voces que arráncan en el ``alzar``. Coincide esta circustancia con que ambas partes tienen dos de sus tres coplas arrando en sílaba tónica.

\todo{texto sobre lo famoso que era la expresión Más valéis vos, Antona} 

\todo{texto sobre Bocangel} 



