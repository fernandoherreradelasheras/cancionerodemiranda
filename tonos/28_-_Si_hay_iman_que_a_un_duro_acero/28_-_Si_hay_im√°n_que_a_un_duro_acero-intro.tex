Del tono \textbf{Si hay imán que a un duro acero} sabemos el título por la entrada correspondiente en el índice del cancionero. Presenta una única sección bipartita con coplas a solo que incluyen una respuesta a cuatro. Al estar escrita la parte solista para alto nos falta más de la mitad del tono sin que exista posibilidad de reconstrucción a partir de las otras partes. La concordancia con el Cancionero poético-musical Hispano de Lisboa, a priori, por faltarle también la parte de alto, debería tener el mismo problema. Sin embargo, las partes convervadas para este tono en el cancionero lisboeta: tiple 1º, alto y tenor. Fíjandonos en el libro del tiple 2º del Cancionero de Miranda vemos que la parte está escrita en clave de Do en 2ª, es decir, para tesitura de alto. Ahí está la explicacicón de por qué en un cancionero se ha conservado y en otro no: se trata de un tono a para un tiple, dos altos y un tenor, así que el segundo alto se copió en el libro del tiple segundo y la fortuna quiso en al menos uno de los cancionero se decidiera que el ato 1º y solista fuera el que habitualmente es el tiple 2ª y no el alto habitual. En resumen, al cancionero de Miranda le falta un alto, y al de Lisboa el otro y por tanto se complementan.

Al superponer las tres voces de la respuesta de una versión con la complementaria de la otra se obeservan pequeñas diferencias en el desarollo polifónico. También habría algunas diferencias mínimas en la parte solita, deducido esto, algún acorde diferente en el acompañamiento que si conservamos. Para esta edición tomamos la voz del alto solista del Cancionero poético-musical Hispano de Lisboa junto con el acompañamiento de la parte a solo. En la respuesta a cuatro se realizan los ajustes necesarios para que la voz incorporada encaje con el material existente respetando así su caracter diferencial con respecto a la otra versión.


