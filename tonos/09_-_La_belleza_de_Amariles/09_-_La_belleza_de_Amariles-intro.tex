A mitad del bloque de los 22 tonos sin estribillo del Cancionero de Miranda, rodeado de tonos con textos atribuidos a poetas portugueses aparece \textbf{La belleza de Amariles}. Aunque el autor de su texto, Miguel de Barrios, nació en Motilla (Córdoba), su familia era de origen portugués. Barrios incluyó el romance La belleza de Amarilis en su \textbf{Flor de Apolo}, cuya primera edición vio la luz en Bruselas en 1665. La versión de tono cambia el nombre a la forma más infrecuente \textit{Amariles}.
