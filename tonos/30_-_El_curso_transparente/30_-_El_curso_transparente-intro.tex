El tono \textit{El curso transparente} aparece cantado en la comedia de 1673 \textit{Los Juegos Olímpicos} escrita por \textbf{Agustín Salazar y Torres}. La música de este tono a cuatro del Cancionero de Miranda no es la misma que la del tono a solo \textbf{Juan Hidalgo}. El de Hidalgo tuvo que ver bastante popular en su tiempo, a tenor de la cantidad de testimonnios musicales diferentes que nos han llegado:
\begin{itemize}
	\item Manuscrito de la Biblioteca de Catalunya E-Bc M 759/15, con acompañamiento para guitarra en cifra.
	\item Manuscrito de la Biblioteca de Catalunya E-Bc M 775/77 (incluye dos copias de las coplas) 
	\item En los folios 82v y 83r del Manuscrito Guerra, E-SCu Ms. 265r
	\item Al inicio del primer cuadernillo de la Colección Harrach de la Biblioteca pública de New York, US-NYp JOG 72-29 vol.1
	\item Copiado por Miguel Gómez Camargo en el borrador de los villancicos para la Navidad de 1689 de la catedral de valladodid E-V 71/45
	\item Manuscrito de la Catedral de Burgos E-BU 61/22
\end{itemize}

Existe otra versión musical incompleta copiada en el folio 113r del manuscrito Gayangos-Barbieri que si bien recuerda en sus ocho compases a la melodía de Hidalgo, las diferencias son suficientes como para considerarlo el bosquejo de una obra diferente. Son también testimonio de la existencia del tono cantado su presencia en obras poéticas que recopilaban letras de tonos cantados como son el \textit{Libro de tonos de Jeronimo Nieto} o la segunda parte del Libro de tonos puestos en cifra para arpa E-Mn M/2478. Este último manuscrito incluye también una parodia en clave escatológica de las coplas originales: \textit{El culo transparente // de tu ceñida braga}.

El tono a 4 del Cancionero de Miranda omite una de las coplas incluidas en la versión de Hidalgo (la cuarta). Para el primer verso del estribillo, \textit{peces, fieras, aves}, las tres primeras voces cantan cada una una palabra diferente al unísono mientras el tenor permanece en silencio.

