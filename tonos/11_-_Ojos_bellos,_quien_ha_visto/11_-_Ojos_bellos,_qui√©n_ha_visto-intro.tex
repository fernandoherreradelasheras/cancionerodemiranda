
El tono \textit{Ojos bellos, quién ha visto} pone en música otro de los romances en castellano de Frei António das Chagas (António da Fonseca Soares) que permanecen inéditos. El texto se conserva en al menos en una fuente manuscrita: \textit{Romances portugueses e castelhanos que compôs Frei António das Chagas antes de ser religioso}, P-Lant PT/TT/MSLIV/1726\footnote{\textsuperscript{}Disponible en https://digitarq.arquivos.pt/details?id=4248770} (p. 452-3). El manuscrito \textit{Poesías Varias, Portuguesas y Castellanas. Obras del insigne Fr. Antonio das Chagas} GB-Lbl Egerton 660 parece contener también una versión de este romance según la descripción realizada por Pascual Gayangos en su primer volumen del \textit{Catalogue of the manuscritps in the Spanish language in the British Museum} de 1875 concuerdan el íncipit y el antetítulo ("A unos ojos negros"). No ha sido posible consultar este manuscrito puesto que, a fecha de hoy (Enero 2025), la British Library aún no ha reestablecido los servicios de consulta y reproducción de manuscritos tras el ciberataque recibido en Octubre de 2023.

La versión del texto del tono y la del manuscrito poético del Arquivo Torre do Tombo concuerdan, con alguna diferencia menor, en las tres primeras cuartetas siendo las restantes, dos en el caso del tono y tres en el caso del la compilación poética, diferentes. 
