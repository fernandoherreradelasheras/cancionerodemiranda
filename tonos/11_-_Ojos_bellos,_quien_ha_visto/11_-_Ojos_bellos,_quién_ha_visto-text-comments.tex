\noindent \textbf{2}: la versión del verso en PT-TT-MSLIV-1726 \textit{más que en vuestras luces negras} parece deberse a un error de copia pues presenta una sílaba de más\\
\textbf{6}: mantenemos \textit{mesmas} para conservar la rima\\
\textbf{10}: ``muestra`` en PT-TT-MSLIV-1726 en vez de ``deja``\\
\textbf{11}: PT-TT-MSLIV-1726 usa la voz portuguesa ``acebiche``\\
\textbf{13}: ``de luto`` en el manuscrito. Tomamos la versión de la fuente de la Biblioteca de Ajuda que omite la preposición para mantener el octosílabo\\
\textbf{17}: esta última cuarteta no está copiada en la parte del tiple 2º\\
\textbf{20}: ``firmezas`` en el manuscrito. Tomamos la lectura del manuscrito de la Biblioteca de Ajuda que tiene más sentido.
